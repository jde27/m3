\usepackage{amsmath,amscd,amssymb,amsthm,enumerate,verbatim,palatino,parskip,graphicx}
\usepackage{amsbsy,anysize,afterpage,framed,cancel}
\marginsize{2.3cm}{2.3cm}{2cm}{2.4cm}
\usepackage{enumerate}

\renewcommand\vec[1]{\ensuremath\boldsymbol{#1}}
\usepackage[T1]{fontenc}
\usepackage{hyperref}


\newcommand{\dd}[2]{\frac{d #1}{d #2}}
\newcommand{\pd}[2]{\frac{\partial #1}{\partial #2}}
\newcommand{\ppd}[2]{\frac{\partial^2 #1}{\partial #2 ^2}}
\newcommand{\ppdd}[3]{\frac{\partial^2 #1}{\partial #2\partial #3}}
\newcommand{\brf}[2]{\left(\frac{#1}{#2}\right)}
                       % Bracket-frac, e.g. for (n\pi x/L) in Fourier series
\newcommand{\fsin}[1]{\sin\brf{#1 \pi x}{L}}
\newcommand{\fcos}[1]{\cos\brf{#1 \pi x}{L}}

\newcommand{\HRule}{\rule{\linewidth}{0.5mm}}

\newcommand{\matr}[4]{\left(\begin{array}{cc}#1&#2\\#3&#4\end{array}\right)}
\newcommand{\twob}[2]{\left(\begin{array}{c}#1\\#2\end{array}\right)}

\newcommand{\m}{\mathcal}
\newcommand{\BB}{\mathbf}
\newcommand{\CC}{\mathbf{C}}
\newcommand{\RR}{\mathbf{R}}
\newcommand{\QQ}{\mathbf{Q}}
\newcommand{\PP}{\mathbf{P}}
\newcommand{\ZZ}{\mathbf{Z}}
\newcommand{\HH}{\mathbf{H}}
\newcommand{\KK}{\mathbf{K}}
\newcommand{\OP}{\operatorname}
\newcommand{\op}{\operatorname}
\newcommand{\into}{\hookrightarrow}
\newcommand{\Hom}{\mathrm{Hom}}
\newcommand{\End}{\OP{End}}
\newcommand{\ad}{\OP{ad}}
\newcommand{\Ad}{\OP{Ad}}
\newcommand{\Sym}{\OP{Sym}}
\renewcommand{\sl}{\mathfrak{sl}}
\renewcommand{\gg}{\mathfrak{g}}
\newcommand{\hh}{\mathfrak{h}}
\newcommand{\gl}{\mathfrak{gl}}
\newcommand{\id}{\mathrm{id}}
\newcommand{\diag}{\mathrm{diag}}
\newcommand{\Tr}{\mathrm{Tr}}

\setcounter{secnumdepth}{1}

\newtheorem{thm}{Theorem}[chapter]
\newtheorem{thmalpha}{Theorem}
\newtheorem{lma}[thm]{Lemma}
\newtheorem{lmaclub}[thm]{$\clubsuit$ Lemma}
\newtheorem{prp}[thm]{Proposition}
\newtheorem{cor}[thm]{Corollary}
\newtheorem{prb}{Problem}

\theoremstyle{definition}

\newtheorem{dfn}[thm]{Definition}
\newtheorem{exm}[thm]{Example}
\newtheorem{exmclub}[thm]{$\clubsuit$ Example}
\newtheorem{obs}[thm]{Observation}
\newtheorem*{clm}{Claim}

\newtheoremstyle{check}% name of the style to be used
  {}% measure of space to leave above the theorem. E.g.: 3pt
  {}% measure of space to leave below the theorem. E.g.: 3pt
  {}% name of font to use in the body of the theorem
  {}% measure of space to indent
  {\bf}% name of head font
  {!}% punctuation between head and body
  { }% space after theorem head; " " = normal interword space
  {}% manually specify head
\theoremstyle{check}
\newtheorem*{chk}{Check}

\theoremstyle{remark}
\newtheorem{rmk}[thm]{Remark}

\setcounter{tocdepth}{1}

\newtheoremstyle{TheoremNum}
    {\topsep}{\topsep}              %%% space between body and thm
    {\itshape}                      %%% Thm body font
    {}                              %%% Indent amount (empty = no indent)
    {\bfseries}                     %%% Thm head font
    {.}                             %%% Punctuation after thm head
    { }                             %%% Space after thm head
    {\thmname{#1}\thmnote{ \bfseries #3}}%%% Thm head spec
\theoremstyle{TheoremNum}

\begingroup 
\makeatletter 
\@for\theoremstyle:=definition,remark,plain,check,TheoremNum\do{% 
\expandafter\g@addto@macro\csname th@\theoremstyle\endcsname{% 
\addtolength\thm@preskip\parskip 
}% 
} 
\endgroup 

\renewcommand*{\thethmalpha}{\Alph{thmalpha}}

% Next bit redefines the proof environment so it's more like Arnold's book ``Mathematical methods of classical mechanics''.

\makeatletter \renewenvironment{proof}[1][\proofname]
{\par\pushQED{\qed}\normalfont\topsep6\p@\@plus6\p@\relax\begin{list}{}{\rightmargin=2em\leftmargin=2em}\item[\hskip\labelsep\bfseries#1\@addpunct{.}]\ignorespaces\footnotesize}{\popQED\end{list}\@endpefalse}
\makeatother

\title{Mathematical Methods 3}
\author{Jonny Evans}



%%%%%%%%%%%%%%%%%% Add extra space before theorems

\begingroup 
\makeatletter 
\@for\theoremstyle:=definition,remark,plain,TheoremNum\do{% 
\expandafter\g@addto@macro\csname th@\theoremstyle\endcsname{% 
\addtolength\thm@preskip\parskip 
}% 
} 
\endgroup 
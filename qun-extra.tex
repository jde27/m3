\documentclass[12pt]{article}

\usepackage{parskip,amsthm,amsmath,amsfonts,amssymb}
\usepackage{multicol,cancel}
\usepackage[shortlabels]{enumitem}
\usepackage[letterpaper,margin=1in,bottom=0.7in]{geometry}
\newcommand{\dd}[2]{\dfrac{d #1}{d #2}}
\newcommand{\ddd}[2]{\dfrac{d^2 #1}{d #2^2}}
\newcommand{\pd}[2]{\dfrac{\partial #1}{\partial #2}}
\newcommand{\ppd}[2]{\dfrac{\partial^2 #1}{\partial #2^2}}
\newcommand{\ppdd}[3]{\dfrac{\partial^2 #1}{\partial #2\partial #3}}
\newcommand{\brf}[2]{\left(\frac{#1}{#2}\right)}
                       % Bracket-frac, e.g. for (n\pi x/L) in Fourier series
\newcommand{\fsin}[1]{\sin\brf{#1 \pi x}{L}}
\newcommand{\fcos}[1]{\cos\brf{#1 \pi x}{L}}
\newcommand{\fsint}[1]{\sin\brf{#1 \pi t}{L}}
\newcommand{\fcost}[1]{\cos\brf{#1 \pi t}{L}}
\newcommand{\RR}{\mathbf{R}}
\newcommand{\CC}{\mathbf{C}}
\newcommand{\ZZ}{\mathbf{Z}}
\newcommand{\mks}[1]{\begin{flushright}(#1 marks)\end{flushright}}

\usepackage{tikz}
\newcommand{\lapl}[6]{
\begin{tikzpicture}[scale=2]
  \draw (0,0) -- (0,1);
  \draw (0,1) -- (1,1);
  \draw (1,1) -- (1,0);
  \draw (1,0) -- (0,0);
  \node at (0.5,0.5) {$#6$};
  \node [below] at (0.5,0) {$#1(x,0)=#2$};
  \node [above] at (0.5,1) {$#1(x,\pi)=#3$};
  \node [left] at (0,0.5) {$#1(0,y)=#4$};
  \node [right] at (1,0.5) {$#1(\pi,y)=#5$};
\end{tikzpicture}
}

\newcommand{\onelapl}[6]{
\begin{tikzpicture}[scale=2]
  \draw (0,0) -- (0,1);
  \draw (0,1) -- (1,1);
  \draw (1,1) -- (1,0);
  \draw (1,0) -- (0,0);
  \node at (0.5,0.5) {#6};
  \node [below] at (0.5,0) {$#1(x,0)=#2$};
  \node [above] at (0.5,1) {$#1(x,1)=#3$};
  \node [left] at (0,0.5) {$#1(0,y)=#4$};
  \node [right] at (1,0.5) {$#1(1,y)=#5$};
\end{tikzpicture}
}


\newcommand{\mlapl}[6]{
\begin{tikzpicture}[scale=2]
  \draw (0,0) -- (0,1);
  \draw (0,1) -- (1,1);
  \draw (1,1) -- (1,0);
  \draw (1,0) -- (0,0);
  \node at (0.5,0.5) {#1};
  \node [below] at (0.5,0) {$#6(x,0)=#2$};
  \node [above] at (0.5,1) {$#6(x,\pi)=#3$};
  \node [left] at (0,0.5) {$#6(0,y)=#4$};
  \node [right] at (1,0.5) {$#6(\pi,y)=#5$};
\end{tikzpicture}
}
\newcommand{\laplneu}[6]{
\begin{tikzpicture}[scale=2]
  \draw (0,0) -- (0,1);
  \draw (0,1) -- (1,1);
  \draw (1,1) -- (1,0);
  \draw (1,0) -- (0,0);
  \node at (0.5,0.5) {#6};
  \node [below] at (0.5,0) {$#1(x,0)=#2$};
  \node [above] at (0.5,1) {$#1(x,\pi)=#3$};
  \node [left] at (0,0.5) {$\partial_x#1(0,y)=#4$};
  \node [right] at (1,0.5) {$\partial_x#1(\pi,y)=#5$};
\end{tikzpicture}
}

\theoremstyle{remark}
\newtheorem{rmk}{Remark}

\theoremstyle{definition}
\newtheorem{question}{Question}
\newtheorem{answer}{Answer}


%%%%%%%%%%%%%%%%%% Add extra space before theorems

\begingroup 
\makeatletter 
\@for\theoremstyle:=definition,remark,plain,TheoremNum\do{% 
\expandafter\g@addto@macro\csname th@\theoremstyle\endcsname{% 
\addtolength\thm@preskip\parskip 
}% 
} 
\endgroup 

\title{Questions for enthusiasts}
\author{J. Evans}
\date{}

\begin{document}
\maketitle



\begin{question}{\bf (Special case of the maximum principle)}\\
Suppose that $\phi(x,y)$ solves Laplace's equation
\[\ppd{\phi}{x}+\ppd{\phi}{y}=0.\]
A nondegenerate critical point is a critical point where the Hessian
\[
\left(\begin{array}{cc}
\ppd{\phi}{x} & \ppdd{\phi}{x}{y}\\
\ppdd{\phi}{x}{y} & \ppd{\phi}{y}
\end{array}\right)
\]
is an invertible matrix (in particular it cannot have zero eigenvalues). Show that $\phi$ has no nondegenerate maxima or minima.

{\em Hint: Take the trace of the Hessian (the sum of its diagonal values) and recall that the trace does not change under conjugation.}
\end{question}

\iffalse
\begin{answer}
If $(x,y)$ is a nondegenerate critical point then the eigenvalues of the Hessian are nonzero. If $(x,y)$ is a maximum or minimum then the eigenvalues $\lambda_1,\lambda_2$ of the Hessian matrix have the same sign. The trace of a matrix is the sum of its diagonal values and equals the trace of its diagonalisation, therefore
\[\ppd{\phi}{x}+\ppd{\phi}{y}=\lambda_1+\lambda_2.\]
But the sum of the eigenvalues is either positive (if both are positive) or negative (if both are negative) and hence never zero.
\end{answer}
\fi

\bigskip

\begin{question}{\bf (Uniqueness of solutions to Laplace's equation)}\\
Let $U\subset\RR^2$ be an open set with boundary curve $\partial U$. Recall Green's theorem $\int_U\nabla\cdot v dxdy=\int_{\partial U}v\cdot\hat{n}ds$ where $\hat{n}$ is the unit outward normal to $\partial U$ and $ds$ is the length element on $\partial U$. Let $\Delta$ denote the Laplacian operator $\Delta\phi=\ppd{\phi}{x}+\ppd{\phi}{y}$.
\begin{enumerate}[(a)]
\item Suppose that $\phi\colon U\to\RR$ is a function with $\phi(x)=0$ for $x\in\partial U$. By considering $v=\phi\nabla\phi$, show that $\int_U\phi\Delta\phi dxdy=-\int_U|\nabla\phi|^2dxdy$.
\item Deduce that if $\Delta\phi=0$ then $\phi(x)=0$ for all $x\in U$.
\item Deduce that if $\phi_1$ and $\phi_2$ are solutions to Laplace's equation and $\phi_1(x)=\phi_2(x)$ for all $x\in\partial U$ then $\phi_1(x)=\phi_2(x)$ for all $x\in U$.
\end{enumerate}
\end{question}

\newpage

\begin{question}{\bf (Hurwitz's solution to the isoperimetric problem)}\\
Suppose that $\gamma(t)=(x(t),y(t))$ is a path in the plane such that $\gamma(t+2\pi)=\gamma(t)$. Suppose that $\gamma([0,2\pi])$ has length $K$ and that $\gamma$ is parametrised by arc-length, so that $\sqrt{\dot{x}^2+\dot{y}^2}=K/2\pi$. Note that because it is parametrised by arc-length, we have
\[K^2=2\pi\int_0^{2\pi}\left(\dot{x}^2+\dot{y}^2\right)dt.\]
If
\begin{align*}
x(t)&=a_0+\sum_{k=1}^{\infty}\left(a_k\cos(kt)+b_k\sin(kt)\right)\\
y(t)&=c_0+\sum_{k=1}^{\infty}\left(c_k\cos(kt)+d_k\sin(kt)\right)
\end{align*}
are Fourier series for the functions $x(t)$ and $y(t)$, write expressions for $K^2$ and the area
\[A=\int_0^{2\pi}x\dot{y}dt\]
bounded by $\gamma$ in terms of the Fourier coefficients. Use these formulae to deduce the {\em isoperimetric inequality}:
\[K^2-4\pi A\geq 0.\]
When does equality hold?
\end{question}

\bigskip

\begin{question}{\bf (Gram-Schmidt orthogonalisation)}\\
Let $V$ be the vector space of (square-integrable) functions on the interval $[-1,1]$ and let
\[\langle f,g\rangle=\int_{-1}^1f(x)g(x)dx\]
denote the $L^2$-inner product on $V$.

The Gram-Schmidt process in linear algebra starts with a sequence of linearly independent vectors $v_1,v_2,v_3,\ldots$ and produces an orthonormal basis $u_1,u_2,u_3,\ldots$ (normalise the first vector $v_1$ to get $u_1$, then set $u_2$ to be the normalisation of $v_2-(v_2\cdot u_1)u_1$, etc.). One can apply Gram-Schmidt to the sequence of polynomials $1,x,x^2,x^3,\ldots$ in $V$ to produce an basis orthonormal with respect to the $L^2$-inner product. Compute the first few orthonormal polynomials.
\end{question}

\newpage

\begin{question}{\bf (The $\Gamma$-function)}\\
Define $\Gamma(r)=\int_0^{\infty}x^{r-1}e^{-x}dx$ for $r>0$ a positive real number.
\begin{enumerate}[(a)]
\item Prove that $\Gamma(r+1)=r\Gamma(r)$.  We define $\Gamma(r)$ for $r\in(-1,0)$ by setting $\Gamma(r):=\Gamma(r+1)/r$ (and inductively we can extend to all negative non-integers). In particular $\Gamma(-1/2)=2\Gamma(1/2)$.
\item Compute $\Gamma(1)$ and prove that $\Gamma(n+1)=n!$ if $n$ is a positive integer.
\item Show that $\Gamma(1/2)=\sqrt{\pi}$.
\item Compute $\int_0^{\infty}\sqrt{y}e^{-y^3}dy$ and $\int_0^1\frac{dx}{\sqrt{-\ln(x)}}$.
\end{enumerate}
\end{question}

\bigskip

\begin{question}{\bf (Sturm-Liouville systems)}
Let $p(x),q(x)$ be functions and define the differential operator
\[Dy=\dd{}{x}\left(p\dd{y}{x}\right)+qy.\]
An eigenfunction of $D$ with eigenvalue $\lambda$ is a (nonzero, possible complex-valued) function $y$ such that $Dy=\lambda y$.
\begin{enumerate}
\item[(a)] Show that $\langle f,Dg\rangle=\langle Df,g\rangle$.
\item[(b)] Prove that if $\lambda$ is an eigenvalue of $D$ then $\lambda\in\RR$.
\item[(c)] Prove that if $y_i$ is a $\lambda_i$ eigenfunction, for $i=1,2$, and $\lambda_1\neq\lambda_2$ then $y_1$ and $y_2$ are orthogonal.
\end{enumerate}
\end{question}

\bigskip

\begin{question}{\bf (The minimal surface equation)}\\
Show that the Euler-Lagrange equation for the functional
\[\int_0^1\int_0^1\sqrt{1+\left(\pd{\phi}{x}\right)^2+\left(\pd{\phi}{y}\right)^2}dxdy\]
is the {\em minimal surface equation}
\[\ppd{\phi}{x}\left(1+\left(\pd{\phi}{y}\right)^2\right)+\ppd{\phi}{y}\left(1+\left(\pd{\phi}{x}\right)^2\right)=2\ppdd{\phi}{x}{y}\pd{\phi}{x}\pd{\phi}{y}.\]
\end{question}

\newpage

\begin{question}\ \\
Consider the Schr\"{o}dinger equation with potential $V(x)$
\[-i\hbar\frac{\partial\psi}{\partial t}=-\frac{\hbar^2}{2m}\frac{\partial^2\psi}{\partial x^2}+V(x)\psi,\quad V(x)=\begin{cases}
1&\mbox{ if }x<0\\
0&\mbox{ if }x\in[0,1]\\
1&\mbox{ if }x>1
\end{cases}\]
for a complex-valued function $\psi$ defined on the whole real line. Recall that the separated solutions $X(x)T(t)$ have $T(t)=e^{iEt/\hbar}$.
\begin{enumerate}
\item[(a)] Assuming that $0<E<1$, show that the general solution for $X$ is
\[X(x)=\begin{cases}
A_1e^{\mu x}+B_1e^{-\mu x}&x\in(-\infty,0]\\
A_2\sin\lambda x+B_2\cos\lambda x&x\in[0,1]\\
A_3e^{\mu x}+B_3e^{-\mu x}&x\in[1,\infty)
\end{cases}\]
where $\mu=\sqrt{2m(1-E)/\hbar^2}$ and $\lambda=\sqrt{2mE/\hbar^2}$.
\item[(b)] We will assume the ``boundary condition at infinity'' that $X$ is bounded as $x\to\infty$. Show that this means $B_1=A_3=0$.
\item[(c)] By further requiring that $X$ and $dX/dx$ are continuous at 0 and at 1, find a system of equations relating $A_1$, $A_2$, $B_2$ and $B_3$.
\item[(d)] Show that these equations imply
\[\cot\left(\frac{2mE}{\hbar^2}\right)=\frac{E-\tfrac{1}{2}}{\sqrt{E(1-E)}}.\]
\item[(e)] Sketch the graph of the solution $X$ you have found (assuming the coefficients are real).
\end{enumerate}
{\em This potential is called the square-well potential. The physical interpretation of $\psi$ is that
\[\int_a^b|\psi(x,t)|^2dx\]
is the probability of finding a particle in the interval $[a,b]$ at time $t$. Since $\psi$ is nonzero in the region outside $[0,1]$ we see that there is a chance the particle can escape the potential well. Classically, a particle could never escape the square-well potential if its energy was less than the ``potential energy barrier'' (in this case $E<1$). The fact that, quantum mechanically, a particle can escape is a phenomenon called quantum tunnelling. For example, this is the mechanism which allows alpha particles to escape from atomic nuclei in radioactive decay.}
\end{question}

\iffalse
\begin{answer}
\begin{enumerate}
\item[(a)] The separated solutions $\psi(x,t)=X(x)T(t)$ satisfy
\[T=e^{iEt/\hbar}\]
and
\[X''=-2m(E-V)X/\hbar^2\]
In the interval $[0,1]$ the potential vanishes, so the equation is
\[X''=-2mEX/\hbar^2\]
and since we are assuming $E>0$ the coefficient $-2mE/\hbar^2=-\lambda^2$ is negative and the solution is
\[A\sin(x\lambda)+B\cos(x\lambda).\]
In the subset $(-\infty,0)\cup(1,\infty)$ the potential is identically equal to 1, so the equation is
\[X''=-2m(E-1)X/\hbar^2\]
and since we are assuming $E<1$ the coefficient $-2m(E-1)/\hbar^2=\mu^2$ is positive and the solution is
\[Ae^{\mu x}+Be^{-\mu x}.\]
The constants $A$ and $B$ can be chosen differently on each connected interval so as to satisfy the boundary conditions so we label them with an index and write the general solution as
\[X(x)=\begin{cases}
A_1e^{\mu x}+B_1e^{-\mu x}&x\in(-\infty,0]\\
A_2\sin\lambda x+B_2\cos\lambda x&x\in[0,1]\\
A_3e^{\mu x}+B_3e^{\mu x}&x\in[1,\infty)
\end{cases}\]
\item[(b)] The condition that $X$ is bounded at infinity means that $B_1$ must vanish, otherwise the term $B_1e^{-\mu x}$ goes to infinity as $x\to-\infty$. Similarly, $A_3$ must vanish.

\item[(c)] At $x=0$, $X(0)=A_1(=A_1e^{\mu\times 0})$ according to the first part of the solution and $X(0)=B_2(=B_2\cos(0\times\lambda)+A_2\sin(0\times\lambda))$ according to the second. Therefore $B_2=A_1$ for continuity of $X$ at 0. Continuity of $X'$ at $x=0$ gives
\[\mu A_1=\lambda A_2\cos(0\times\lambda)+B_2\sin(0\times\lambda)=\lambda A_2\]
so that $A_2=\mu A_1/\lambda$.

At $x=1$, continuity of $X$ becomes
\[A_2\sin\lambda+B_2\cos\lambda=B_3e^{-\mu}\]
or
\[A_1\left(\frac{\mu}{\lambda}\sin\lambda+\cos\lambda\right)e^{\mu}=B_3\]
and continuity of $X'$ becomes
\[\lambda A_2\cos\lambda-\lambda B_2\sin\lambda=-\mu B_3e^{-\mu}\]
or
\[A_1\left(\cos\lambda-\frac{\lambda}{\mu}\sin\lambda\right)e^{\mu}=-B_3\]
\item[(d)] Together, these imply that
\[\frac{\lambda}{\mu}\sin\lambda-\cos\lambda=\frac{\mu}{\lambda}\sin\lambda+\cos\lambda\]
for consistency (by comparing the two expressions for $B_3e^{-\mu}/A_1$). This gives the relationship
\[2\cot\lambda=\frac{\lambda^2-\mu^2}{\mu\lambda}\]
or, since $\lambda^2=2mE/\hbar^2$ and $\mu^2=2m(1-E)/\hbar^2$,
\[\cot\frac{2mE}{\hbar^2}=\frac{E-\tfrac{1}{2}}{\sqrt{E(1-E)}}.\]
\end{enumerate}
\end{answer}
\fi


\end{document}
\documentclass[12pt]{article}

\usepackage{parskip,amsthm,amsmath,amsfonts,amssymb}
\usepackage{multicol,cancel}
\usepackage[shortlabels]{enumitem}
\usepackage[letterpaper,margin=1in,bottom=0.7in]{geometry}
\newcommand{\dd}[2]{\dfrac{d #1}{d #2}}
\newcommand{\ddd}[2]{\dfrac{d^2 #1}{d #2^2}}
\newcommand{\pd}[2]{\dfrac{\partial #1}{\partial #2}}
\newcommand{\ppd}[2]{\dfrac{\partial^2 #1}{\partial #2^2}}
\newcommand{\ppdd}[3]{\dfrac{\partial^2 #1}{\partial #2\partial #3}}
\newcommand{\brf}[2]{\left(\frac{#1}{#2}\right)}
                       % Bracket-frac, e.g. for (n\pi x/L) in Fourier series
\newcommand{\fsin}[1]{\sin\brf{#1 \pi x}{L}}
\newcommand{\fcos}[1]{\cos\brf{#1 \pi x}{L}}
\newcommand{\fsint}[1]{\sin\brf{#1 \pi t}{L}}
\newcommand{\fcost}[1]{\cos\brf{#1 \pi t}{L}}
\newcommand{\RR}{\mathbf{R}}
\newcommand{\CC}{\mathbf{C}}
\newcommand{\ZZ}{\mathbf{Z}}
\newcommand{\mks}[1]{\begin{flushright}(#1 marks)\end{flushright}}

\usepackage{tikz}
\newcommand{\lapl}[6]{
\begin{tikzpicture}[scale=2]
  \draw (0,0) -- (0,1);
  \draw (0,1) -- (1,1);
  \draw (1,1) -- (1,0);
  \draw (1,0) -- (0,0);
  \node at (0.5,0.5) {$#6$};
  \node [below] at (0.5,0) {$#1(x,0)=#2$};
  \node [above] at (0.5,1) {$#1(x,\pi)=#3$};
  \node [left] at (0,0.5) {$#1(0,y)=#4$};
  \node [right] at (1,0.5) {$#1(\pi,y)=#5$};
\end{tikzpicture}
}

\newcommand{\onelapl}[6]{
\begin{tikzpicture}[scale=2]
  \draw (0,0) -- (0,1);
  \draw (0,1) -- (1,1);
  \draw (1,1) -- (1,0);
  \draw (1,0) -- (0,0);
  \node at (0.5,0.5) {#6};
  \node [below] at (0.5,0) {$#1(x,0)=#2$};
  \node [above] at (0.5,1) {$#1(x,1)=#3$};
  \node [left] at (0,0.5) {$#1(0,y)=#4$};
  \node [right] at (1,0.5) {$#1(1,y)=#5$};
\end{tikzpicture}
}


\newcommand{\mlapl}[6]{
\begin{tikzpicture}[scale=2]
  \draw (0,0) -- (0,1);
  \draw (0,1) -- (1,1);
  \draw (1,1) -- (1,0);
  \draw (1,0) -- (0,0);
  \node at (0.5,0.5) {#1};
  \node [below] at (0.5,0) {$#6(x,0)=#2$};
  \node [above] at (0.5,1) {$#6(x,\pi)=#3$};
  \node [left] at (0,0.5) {$#6(0,y)=#4$};
  \node [right] at (1,0.5) {$#6(\pi,y)=#5$};
\end{tikzpicture}
}
\newcommand{\laplneu}[6]{
\begin{tikzpicture}[scale=2]
  \draw (0,0) -- (0,1);
  \draw (0,1) -- (1,1);
  \draw (1,1) -- (1,0);
  \draw (1,0) -- (0,0);
  \node at (0.5,0.5) {#6};
  \node [below] at (0.5,0) {$#1(x,0)=#2$};
  \node [above] at (0.5,1) {$#1(x,\pi)=#3$};
  \node [left] at (0,0.5) {$\partial_x#1(0,y)=#4$};
  \node [right] at (1,0.5) {$\partial_x#1(\pi,y)=#5$};
\end{tikzpicture}
}

\theoremstyle{remark}
\newtheorem{rmk}{Remark}

\theoremstyle{definition}
\newtheorem{question}{Question}
\newtheorem{answer}{Answer}


%%%%%%%%%%%%%%%%%% Add extra space before theorems

\begingroup 
\makeatletter 
\@for\theoremstyle:=definition,remark,plain,TheoremNum\do{% 
\expandafter\g@addto@macro\csname th@\theoremstyle\endcsname{% 
\addtolength\thm@preskip\parskip 
}% 
} 
\endgroup 

\title{Methods 3 - Question Sheet 0}
\author{J. Evans}
\date{}

\begin{document}
\maketitle

This is intended as a revision sheet: everything on it was covered at A-level or in Methods 1 and 2. I encourage you to look through it and make sure you remember how to do everything before it comes up in lectures.

\begin{question}(Will be useful for Fourier series)\\
Integrate the following functions (in each case $n$ is an integer).
\begin{enumerate}
\item $\int_0^{\pi}x(x-\pi)\sin(nx)dx$.
\item $\int_0^1(\cos(2\pi x)-1)\sin(n\pi x)dx$.
\item $\int_0^{\pi}x\sin(nx)dx$.
\item $\int_0^{\pi}x^3\sin(nx)dx$.
\item $\int_0^{\pi}e^x\sin(nx)dx$.
\end{enumerate}
\end{question}

\bigskip

\begin{question}(Will be useful for separation of variables)\\
What is the general solution of the following second order ODEs? In each case, $k$ is a constant.
\begin{enumerate}[(a)]
\item $y''=-k^2y$,
\item $y''=k^2y$,
\item $y''+y'+k^2y=0$,
\item $y''=y+x$.
\end{enumerate}
\end{question}

\bigskip

\begin{question}(Will be useful for solving Euler-Lagrange equations)\\
Do the following integrals ($C$ is a constant):
\begin{enumerate}
\item[(a)] $\int\frac{du}{\sqrt{u-1}}$,
\item[(b)] $\int\frac{du}{\sqrt{1-u^2}}$
\item[(c)] $\int\frac{dy}{\sqrt{\frac{1}{Cy^2}-1}}$
\item[(d)] $\int\frac{dy}{\sqrt{\frac{1}{-Cy}-1}}$ (Note: This makes sense as long as $-1<Cy<0$).
\item[(e)] $\int\frac{dt}{\sin t\sqrt{\sin^2 t-C^2}}$ ({\em Hint: Make a substitution $u=\cot t$ to put the integrand into a more familiar form.}).
\end{enumerate}
\end{question}

\bigskip

\begin{question}(Will be useful for constrained Euler-Lagrange)\\
Find the critical points of the function
\[F(x,y,z)=x^2+3y^2+z^2\]
subject to the constraints:
\begin{enumerate}
\item[(a)] $x^2+y^2-z^2=2$.
\item[(b)] $x^2+y^2-z=2$, $x^2-y^2-z=3$.
\end{enumerate}
{\em Hint: For (b) use two Lagrange multipliers, one for each constraint.}
\end{question}

\bigskip

{\em The next two questions focus on the chain rule; make sure you are using the version of the chain rule from Methods 2, valid for functions of several variables, rather than the usual one-variable chain rule!}

\bigskip

\begin{question}(Will be useful for Euler-Lagrange and method of characteristics)\\
Let $(u(x,y),v(x,y))$ be a vector-valued function of two variables and let
$(x(t),y(t))$ be a path in the plane. If we consider the values of $(u,v)$
along the path, $(u(x(t),y(t)),v(x(t),y(t)))$, what does the chain rule
tell us that $\dd{u}{t}$ and $\dd{v}{t}$ are? Show that this is equivalent
to the matrix identity:

\[
\left(
   \begin{array}{c}
        \dd{u}{t}\\
        \dd{v}{t}
   \end{array}
\right)   =   \left(
                 \begin{array}{cc}
                      \pd{u}{x} & \pd{u}{y}\\
                      \pd{v}{t} & \pd{v}{y}
                 \end{array}
              \right)                       \left(
                                               \begin{array}{c}
                                                    \dd{x}{t}\\
                                                    \dd{y}{t}
                                               \end{array}
                                            \right)
\]

Find the derivative of $F(x,y)=\sin(xy^2)$ restricted to the path $(x(t),y(t))=(1+t,1-t)$ at the point $t=0$:
\begin{enumerate}
 \item[(a)] directly and
 \item[(b)] using the chain rule.
\end{enumerate}
\end{question}

\bigskip

\begin{question}(D'Alembert's method)\\
Let $c$ be a real number and let $\phi(x,t)$ be a function on $\RR^2$. Define $u$ and $v$ by $u=x+ct$ and $v=x-ct$. Find an expression for $x$ and $t$ in terms of $u$ and $v$ and hence (or otherwise) show that
\[\ppdd{\phi}{u}{v}=\frac{1}{4}\left(\ppd{\phi}{x}-\frac{1}{c^2}\ppd{\phi}{t}\right).\]
If $\phi$ is a solution to $\ppd{\phi}{x}-\dfrac{1}{c^2}\ppd{\phi}{t}$, deduce that there exist functions $F$ and $G$ such that $\phi(x,t)=F(x+ct)+G(x-ct)$.
\end{question}

\end{document}
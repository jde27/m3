\documentclass[12pt]{article}

\usepackage{parskip,amsthm,amsmath,amsfonts,amssymb}
\usepackage{multicol,cancel}
\usepackage[shortlabels]{enumitem}
\usepackage[letterpaper,margin=1in,bottom=0.7in]{geometry}
\newcommand{\dd}[2]{\dfrac{d #1}{d #2}}
\newcommand{\ddd}[2]{\dfrac{d^2 #1}{d #2^2}}
\newcommand{\pd}[2]{\dfrac{\partial #1}{\partial #2}}
\newcommand{\ppd}[2]{\dfrac{\partial^2 #1}{\partial #2^2}}
\newcommand{\ppdd}[3]{\dfrac{\partial^2 #1}{\partial #2\partial #3}}
\newcommand{\brf}[2]{\left(\frac{#1}{#2}\right)}
                       % Bracket-frac, e.g. for (n\pi x/L) in Fourier series
\newcommand{\fsin}[1]{\sin\brf{#1 \pi x}{L}}
\newcommand{\fcos}[1]{\cos\brf{#1 \pi x}{L}}
\newcommand{\fsint}[1]{\sin\brf{#1 \pi t}{L}}
\newcommand{\fcost}[1]{\cos\brf{#1 \pi t}{L}}
\newcommand{\RR}{\mathbf{R}}
\newcommand{\CC}{\mathbf{C}}
\newcommand{\ZZ}{\mathbf{Z}}
\newcommand{\mks}[1]{\begin{flushright}(#1 marks)\end{flushright}}

\usepackage{tikz}
\newcommand{\lapl}[6]{
\begin{tikzpicture}[scale=2]
  \draw (0,0) -- (0,1);
  \draw (0,1) -- (1,1);
  \draw (1,1) -- (1,0);
  \draw (1,0) -- (0,0);
  \node at (0.5,0.5) {$#6$};
  \node [below] at (0.5,0) {$#1(x,0)=#2$};
  \node [above] at (0.5,1) {$#1(x,\pi)=#3$};
  \node [left] at (0,0.5) {$#1(0,y)=#4$};
  \node [right] at (1,0.5) {$#1(\pi,y)=#5$};
\end{tikzpicture}
}

\newcommand{\onelapl}[6]{
\begin{tikzpicture}[scale=2]
  \draw (0,0) -- (0,1);
  \draw (0,1) -- (1,1);
  \draw (1,1) -- (1,0);
  \draw (1,0) -- (0,0);
  \node at (0.5,0.5) {#6};
  \node [below] at (0.5,0) {$#1(x,0)=#2$};
  \node [above] at (0.5,1) {$#1(x,1)=#3$};
  \node [left] at (0,0.5) {$#1(0,y)=#4$};
  \node [right] at (1,0.5) {$#1(1,y)=#5$};
\end{tikzpicture}
}


\newcommand{\mlapl}[6]{
\begin{tikzpicture}[scale=2]
  \draw (0,0) -- (0,1);
  \draw (0,1) -- (1,1);
  \draw (1,1) -- (1,0);
  \draw (1,0) -- (0,0);
  \node at (0.5,0.5) {#1};
  \node [below] at (0.5,0) {$#6(x,0)=#2$};
  \node [above] at (0.5,1) {$#6(x,\pi)=#3$};
  \node [left] at (0,0.5) {$#6(0,y)=#4$};
  \node [right] at (1,0.5) {$#6(\pi,y)=#5$};
\end{tikzpicture}
}
\newcommand{\laplneu}[6]{
\begin{tikzpicture}[scale=2]
  \draw (0,0) -- (0,1);
  \draw (0,1) -- (1,1);
  \draw (1,1) -- (1,0);
  \draw (1,0) -- (0,0);
  \node at (0.5,0.5) {#6};
  \node [below] at (0.5,0) {$#1(x,0)=#2$};
  \node [above] at (0.5,1) {$#1(x,\pi)=#3$};
  \node [left] at (0,0.5) {$\partial_x#1(0,y)=#4$};
  \node [right] at (1,0.5) {$\partial_x#1(\pi,y)=#5$};
\end{tikzpicture}
}

\theoremstyle{remark}
\newtheorem{rmk}{Remark}

\theoremstyle{definition}
\newtheorem{question}{Question}
\newtheorem{answer}{Answer}


%%%%%%%%%%%%%%%%%% Add extra space before theorems

\begingroup 
\makeatletter 
\@for\theoremstyle:=definition,remark,plain,TheoremNum\do{% 
\expandafter\g@addto@macro\csname th@\theoremstyle\endcsname{% 
\addtolength\thm@preskip\parskip 
}% 
} 
\endgroup 


\title{Methods 3 - Question Sheet 6}
\author{J. Evans}
\date{}

\begin{document}
\maketitle

In this sheet $\phi_x$ denotes $\pd{\phi}{x}$.

\begin{question}(10 marks for * parts)\\
Find the Euler-Lagrange equation for the following functionals (we have only written the Lagrangian in each case - if you need to, assume that the domain of integration is $(x,y)\in[0,1]^2$)
\begin{enumerate}[(a)]
\item * $\frac{1}{2}\left(\phi_x^2+\phi_y^2\right)+\frac{1}{2}K\phi^2$ (where $K$ is a constant),
\item * $\phi^2\phi_x^2+\phi_y^2$,
\item * $\phi_x\phi_y$ (in this case, also find the general solution to the Euler-Lagrange equation),
\item $\frac{1}{2}\left(\phi_x^2+\phi_y^2\right)$ subject to the constraint $\int\phi^2 dxdy=K$.
\item $\frac{1}{\phi_x}+\frac{1}{\phi_y}$.
\end{enumerate}
\end{question}

\iffalse
\begin{answer}
\begin{enumerate}[(a)]
\item The Euler-Lagrange equation
\[\pd{}{x}\pd{L}{\phi_x}+\pd{}{y}\pd{L}{\phi_y}=\pd{L}{\phi}\]
becomes
\[\pd{}{x}\pd{\phi}{x}+\pd{}{y}\pd{\phi}{y}=K\phi\]
that is the Helmholtz equation $\Delta\phi=K\phi$.\mks{3}
\item The Euler-Lagrange equation
\[\pd{}{x}\pd{L}{\phi_x}+\pd{}{y}\pd{L}{\phi_y}=\pd{L}{\phi}\]
becomes
\[\pd{}{x}\left(2\phi^2\pd{\phi}{x}\right)+\pd{}{y}\left(2\pd{\phi}{y}\right)=2\phi\left(\pd{\phi}{x}\right)^2\]
or
\[4\phi\left(\pd{\phi}{x}\right)^2+2\phi^2\ppd{\phi}{x}+2\ppd{\phi}{y}=2\phi\left(\pd{\phi}{x}\right)^2.\]
Simplifying gives\mks{3}
\[0=\phi^2\ppd{\phi}{x}+\ppd{\phi}{y}+\phi\left(\pd{\phi}{x}\right)^2.\]
\item The Euler-Lagrange equation
\[\pd{}{x}\pd{L}{\phi_x}+\pd{}{y}\pd{L}{\phi_y}=\pd{L}{\phi}\]
becomes
\[\pd{}{x}\pd{\phi}{y}+\pd{}{y}\pd{\phi}{x}=0\]
or
\[\ppdd{\phi}{x}{y}=0.\]
The general solution is $F(x)+G(y)$ for arbitrary functions $F$ and $G$.\mks{4}
\item Modify the functional to
\[\frac{1}{2}\left(\left(\pd{\phi}{x}\right)^2+\left(\pd{\phi}{y}\right)^2\right)-\lambda(\phi^2-K)\]
the Euler-Lagrange equation
\[\pd{}{x}\pd{L}{\phi_x}+\pd{}{y}\pd{L}{\phi_y}=\pd{L}{\phi}\]
becomes
\[\pd{}{x}\pd{\phi}{x}+\pd{}{y}\pd{\phi}{y}=-2\lambda\phi\]
or $\Delta\phi=-2\lambda\phi$ (again this is Helmholtz's equation).
\item The Euler-Lagrange equation
\[\pd{}{x}\pd{L}{\phi_x}+\pd{}{y}\pd{L}{\phi_y}=\pd{L}{\phi}\]
becomes
\[\pd{}{x}\left(-\frac{1}{\phi_x^2}\right)+\pd{}{y}\left(-\frac{1}{\phi_y^2}\right)=0\]
or
\[\frac{2\phi_{xx}}{\phi_x^3}+\frac{2\phi_{yy}}{\phi_y^3}=0.\]
Equivalently
\[\phi_y^3\phi_{xx}+\phi_x^3\phi_{yy}=0.\]
\end{enumerate}
\end{answer}
\fi

\bigskip

\begin{question}\ \\
A string has its endpoints fixed at $(0,0)$ and $(L,0)$. If its height at $x$ and time $t$ is $\phi(x,t)$ then (to a good approximation) its total kinetic energy is $E(\phi)=\frac{\rho}{2}\int_0^L\phi_t^2dx$ and its total potential energy (coming from stretching tension) is $T(\phi)=\frac{\tau}{2}\int_0^1\phi_x^2dx$. The string moves to minimise the integral
\[\int_0^1(E(\phi)-T(\phi))dt.\]
Show that the string obeys the {\em wave equation}
\[\frac{1}{c^2}\frac{\partial^2\phi}{\partial t^2}=\frac{\partial^2\phi}{\partial x^2}\]
where $c=\sqrt{\tau/\rho}$.
\end{question}

\iffalse
\begin{answer}
The Euler-Lagrange equation for $\int_0^1\int_0^L(\frac{\rho}{2}(\partial_t\phi)^2-\frac{\tau}{2}(\partial_x\phi)^2)dxdt$ is
\[\frac{\rho}{2}\frac{\partial}{\partial t}\partial_t\phi-\frac{\tau}{2}\frac{\partial}{\partial x}\partial_x\phi=0\]
or
\[\frac{\rho}{\tau}\partial_t^2\phi=\partial_x^2\phi\]
which is the wave equation.
\end{answer}
\fi

\newpage

\begin{question}(10 marks)\\
\begin{enumerate}[(a)]
\item Find the half-range sine series of the function
\[F(x)=\begin{cases}
x^2&\mbox{ if }x\in[0,1/2]\\
(x-1)^2&\mbox{ if }x\in[1/2,1]
\end{cases}\]

\item Derive the Euler-Lagrange equation for the following functional
\[\int_0^1\int_0^1\left(\left(\pd{\phi}{x}\right)^2+\left(\pd{\phi}{y}\right)^2\right)dxdy\]

\item Solve this Euler-Lagrange equation given the boundary values

\begin{center}
\onelapl{\phi}{0}{0}{F(y)}{0}{}
\end{center}
where $F$ is defined in part (a) of the question.
\end{enumerate}
\end{question}

\iffalse
\begin{answer}
\begin{enumerate}
\item[(a)] We need to calculate the $n$th Fourier coefficient $F_n=2\int_0^1F(x)\sin(n\pi x)dx$. We split the integral as
\[2\int_0^{1/2}x^2\sin(n\pi x)dx+2\int_{1/2}^1(x-1)^2\sin(n\pi x)dx\]
The first gives
\begin{align*}
\int_0^{1/2}x^2\sin(n\pi x)dx&=\left[-x^2\frac{\cos(n\pi x)}{n\pi}\right]_0^{1/2}+2\int_0^{1/2}x\frac{\cos(n\pi x)}{n\pi}dx\\
                            &=\frac{-\cos(n\pi/2)}{4n\pi}+\frac{2}{n\pi}\left[x\frac{\sin(n\pi x)}{n\pi}\right]_0^{1/2}-\frac{2}{n\pi}\int_0^{1/2}\frac{\sin(n\pi x)}{n\pi}dx\\
                            &=\frac{-\cos(n\pi/2)}{4n\pi}+\frac{\sin(n\pi/2)}{n^2\pi^2}+\frac{2}{n^3\pi^3}\left(\cos(n\pi/2)-1\right)
\end{align*}
the second gives
\begin{align*}
\int_{1/2}^1(x-1)^2\sin(n\pi x)dx&=\left[-(x-1)^2\frac{\cos(n\pi x)}{n\pi}\right]_{1/2}^1+2\int_{1/2}^1(x-1)\frac{\cos(n\pi x)}{n\pi}dx\\
                            &=\frac{\cos(n\pi/2)}{4n\pi}+\frac{2}{n\pi}\left[(x-1)\frac{\sin(n\pi x)}{n\pi}\right]_{1/2}^1-\frac{2}{n\pi}\int_{1/2}^1\frac{\sin(n\pi x)}{n\pi}dx\\
                            &=\frac{\cos(n\pi/2)}{4n\pi}+\frac{\sin(n\pi/2)}{n^2\pi^2}+\frac{2}{n^3\pi^3}\left((-1)^n-\cos(n\pi/2)\right)
\end{align*}
Adding these two integrals (and reinserting the factor of 2) gives\mks{6}
\[F_n=\frac{4}{n^2\pi^2}\left(\sin(n\pi/2)+\frac{(-1)^n-1)}{n\pi}\right)\]
so
\[F(x)=\sum_{n=1}^{\infty}\frac{4}{n^2\pi^2}\left(\sin(n\pi/2)+\frac{(-1)^n-1)}{n\pi}\right)\sin(n\pi x).\]
\item[(b)] The Euler-Lagrange equation
\[\pd{}{x}\pd{L}{\partial_x\phi}+\pd{}{y}\pd{L}{\partial_y\phi}=0\]
is
\[\pd{}{x}2\phi_x+\pd{}{y}2\phi_y=0\]
or
\[\ppd{\phi}{x}+\ppd{\phi}{y}=0.\]
This is just Laplace's equation.\mks{1}
\item[(c)] So we need to solve Laplace's equation with the given boundary conditions. Since all the corner values vanish we can make the Ansatz
\[\phi(x,y)=\sum_{n=1}^{\infty}A_n\sin(n\pi y)\sinh(n\pi(1-x))\]
and deduce from $\phi(0,y)=F(y)$ that
\[\sum_{n=1}^{\infty}A_n\sin(n\pi y)\sinh(n\pi)=\sum_{n=1}^{\infty}F_nsin(n\pi y)\]
and so $A_n$ is $F_n\sinh(n\pi)$. Therefore the solution is\mks{3}
\[\sum_{n=1}^{\infty}\frac{4}{n^2\pi^2\sinh(n\pi)}\left(\sin(n\pi/2)+\frac{(-1)^n-1)}{n\pi}\right)\sin(n\pi y)\sinh(n\pi(1-x))\]
\end{enumerate}
\end{answer}
\newpage
\fi

\begin{question}\ \\
Show that the Euler-Lagrange equation for the functional
\[\int_0^1\int_0^1\sqrt{1+\left(\pd{\phi}{x}\right)^2+\left(\pd{\phi}{y}\right)^2}dxdy\]
is the {\em minimal surface equation}
\[\ppd{\phi}{x}\left(1+\left(\pd{\phi}{y}\right)^2\right)+\ppd{\phi}{y}\left(1+\left(\pd{\phi}{x}\right)^2\right)=2\ppdd{\phi}{x}{y}\pd{\phi}{x}\pd{\phi}{y}.\]
\end{question}

\iffalse
\begin{answer}
We will write $\phi_x$ for $\partial_x\phi$ and $j=\sqrt{1+\phi_x^2+\phi_y^2}$ for brevity. The Euler-Lagrange equation is
\[\frac{\partial}{\partial x}\frac{\phi_x}{j}+\frac{\partial}{\partial y}\frac{\phi_y}{j}=0\]
We have $\pd{}{x}\dfrac{\phi_x}{j}=\dfrac{\phi_{xx}}{j}-\dfrac{\phi_x}{j^2}\pd{j}{x}$ and $\pd{j}{x}=\dfrac{\phi_x\phi_{xx}+\phi_y\phi_{yx}}{j}$ (with a similar equation for the second term). Therefore the Euler-Lagrange equation implies
\[
0=\frac{\phi_{xx}+\phi_{yy}}{j}-\frac{1}{j^3}\left(\phi_x^2\phi_{xx}+2\phi_x\phi_y\phi_{xy}+\phi_y^2\phi_{yy}\right)
\]
or
\[0=\phi_{xx}(1+\phi_y^2)+\phi_{yy}(1+\phi_x^2)-2\phi_x\phi_y\phi_{xy}.\]
\end{answer}
\newpage
\fi

\bigskip

\begin{question}
Check that the function $\phi(x,y)=\sqrt{1-x^2-y^2}$ satisfies the {\em constant mean curvature equation}
\[-\lambda=\pd{}{x}\left(\frac{\phi_x}{\sqrt{1+\phi_x^2+\phi_y^2}}\right)+\pd{}{y}\left(\frac{\phi_y}{\sqrt{1+\phi_x^2+\phi_y^2}}\right)\]
for a suitable value of $\lambda$. This proves that hemispherical soap bubbles can exist!
\end{question}


\end{document}
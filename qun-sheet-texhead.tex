\usepackage{parskip,amsthm,amsmath,amsfonts,amssymb}
\usepackage{multicol,cancel}
\usepackage[shortlabels]{enumitem}
\usepackage[letterpaper,margin=1in,bottom=0.7in]{geometry}
\newcommand{\dd}[2]{\dfrac{d #1}{d #2}}
\newcommand{\ddd}[2]{\dfrac{d^2 #1}{d #2^2}}
\newcommand{\pd}[2]{\dfrac{\partial #1}{\partial #2}}
\newcommand{\ppd}[2]{\dfrac{\partial^2 #1}{\partial #2^2}}
\newcommand{\ppdd}[3]{\dfrac{\partial^2 #1}{\partial #2\partial #3}}
\newcommand{\brf}[2]{\left(\frac{#1}{#2}\right)}
                       % Bracket-frac, e.g. for (n\pi x/L) in Fourier series
\newcommand{\fsin}[1]{\sin\brf{#1 \pi x}{L}}
\newcommand{\fcos}[1]{\cos\brf{#1 \pi x}{L}}
\newcommand{\fsint}[1]{\sin\brf{#1 \pi t}{L}}
\newcommand{\fcost}[1]{\cos\brf{#1 \pi t}{L}}
\newcommand{\RR}{\mathbf{R}}
\newcommand{\CC}{\mathbf{C}}
\newcommand{\ZZ}{\mathbf{Z}}
\newcommand{\mks}[1]{\begin{flushright}(#1 marks)\end{flushright}}

\usepackage{tikz}
\newcommand{\lapl}[6]{
\begin{tikzpicture}[scale=2]
  \draw (0,0) -- (0,1);
  \draw (0,1) -- (1,1);
  \draw (1,1) -- (1,0);
  \draw (1,0) -- (0,0);
  \node at (0.5,0.5) {$#6$};
  \node [below] at (0.5,0) {$#1(x,0)=#2$};
  \node [above] at (0.5,1) {$#1(x,\pi)=#3$};
  \node [left] at (0,0.5) {$#1(0,y)=#4$};
  \node [right] at (1,0.5) {$#1(\pi,y)=#5$};
\end{tikzpicture}
}

\newcommand{\onelapl}[6]{
\begin{tikzpicture}[scale=2]
  \draw (0,0) -- (0,1);
  \draw (0,1) -- (1,1);
  \draw (1,1) -- (1,0);
  \draw (1,0) -- (0,0);
  \node at (0.5,0.5) {#6};
  \node [below] at (0.5,0) {$#1(x,0)=#2$};
  \node [above] at (0.5,1) {$#1(x,1)=#3$};
  \node [left] at (0,0.5) {$#1(0,y)=#4$};
  \node [right] at (1,0.5) {$#1(1,y)=#5$};
\end{tikzpicture}
}


\newcommand{\mlapl}[6]{
\begin{tikzpicture}[scale=2]
  \draw (0,0) -- (0,1);
  \draw (0,1) -- (1,1);
  \draw (1,1) -- (1,0);
  \draw (1,0) -- (0,0);
  \node at (0.5,0.5) {#1};
  \node [below] at (0.5,0) {$#6(x,0)=#2$};
  \node [above] at (0.5,1) {$#6(x,\pi)=#3$};
  \node [left] at (0,0.5) {$#6(0,y)=#4$};
  \node [right] at (1,0.5) {$#6(\pi,y)=#5$};
\end{tikzpicture}
}
\newcommand{\laplneu}[6]{
\begin{tikzpicture}[scale=2]
  \draw (0,0) -- (0,1);
  \draw (0,1) -- (1,1);
  \draw (1,1) -- (1,0);
  \draw (1,0) -- (0,0);
  \node at (0.5,0.5) {#6};
  \node [below] at (0.5,0) {$#1(x,0)=#2$};
  \node [above] at (0.5,1) {$#1(x,\pi)=#3$};
  \node [left] at (0,0.5) {$\partial_x#1(0,y)=#4$};
  \node [right] at (1,0.5) {$\partial_x#1(\pi,y)=#5$};
\end{tikzpicture}
}

\theoremstyle{remark}
\newtheorem{rmk}{Remark}

\theoremstyle{definition}
\newtheorem{question}{Question}
\newtheorem{answer}{Answer}


%%%%%%%%%%%%%%%%%% Add extra space before theorems

\begingroup 
\makeatletter 
\@for\theoremstyle:=definition,remark,plain,TheoremNum\do{% 
\expandafter\g@addto@macro\csname th@\theoremstyle\endcsname{% 
\addtolength\thm@preskip\parskip 
}% 
} 
\endgroup 
